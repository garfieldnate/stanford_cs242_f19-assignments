
\documentclass[11pt]{article}

\input{../../tex/defs.tex}

% Useful syntax commands
\newcommand{\jarr}[1]{\left[#1\right]}   % \jarr{x: y} = {x: y}
\newcommand{\jobj}[1]{\left\{#1\right\}} % \jobj{1, 2} = [1, 2]
\newcommand{\pgt}[1]{\, > {#1}}          % \pgt{1} = > 1
\newcommand{\plt}[1]{\, < {#1}}          % \plt{2} = < 2
\newcommand{\peq}[1]{\, = {#1}}          % \peq{3} = = 3
\newcommand{\prop}[1]{\langle{#1}\rangle}% \prop{x} = <x>
\newcommand{\matches}[2]{{#1}\sim{#2}}   % \matches{a}{b} = a ~ b
\newcommand{\aeps}{\varepsilon}          % \apes = epsilon
\newcommand{\akey}[2]{.{#1}\,{#2}}       % \akey{s}{a} = .s a
\newcommand{\aidx}[2]{[#1]\,{#2}}        % \aidx{i}{a} = [i] a
\newcommand{\apipe}[1]{\mid {#1}}        % \apipe{a} = | a

% Other useful syntax commands:
%
% \msf{x} = x (not italicised)
% \falset = false
% \truet = true
% \tnum = num
% \tbool = bool
% \tstr = str


\begin{document}

\hwtitle
  {Assignment 1}
  {Nathan Glenn}

\problem{Problem 1}

Part 1:

Below, we take $n$ to be any numeric value and $s$ to be any string value surrounded by single quotes.

\begin{alignat*}{1}
\msf{Property}~p ::= \qamp \varepsilon \\
\mid \qamp Expr \\
\mid \qamp AndExpr \\
\mid \qamp Expr \lor AndExpr \\
\mid \qamp (AndExpr) \lor AndExpr \\
\\
\msf{And{\mhyphen}Expression}~AndExpr ::= \qamp Expr \land Expr \\
\mid \qamp Expr \land p \\
\\
\msf{Expression}~Expr ::= \qamp \pgt{n} \\
\mid \qamp \plt{n} \\
\mid \qamp \peq{n} \\
\mid \qamp \peq{s} \\
\\
\msf{Schema}~\tau ::= \qamp \tnum \prop{p} \\
\mid \qamp \tstr \prop{p} \\
\mid \qamp \truet \\
\mid \qamp \falset \\
\mid \qamp \tbool \\
\mid \qamp \jobj{} \\
\mid \qamp \jobj{kvl} \\
\mid \qamp \jarr{\tau} \\
\\
\msf{Key{\mhyphen}Value\ List}~kvl ::= \qamp kv \\
\mid \qamp kv , kvl \\
\\
\msf{Key{\mhyphen}Value}~kv ::= s : \tau
\end{alignat*}

Part 2:

% mathpar is the environment for writing inference rules. It takes care of
% the spacing and line breaks automatically. Use "\\" in the premises to
% space out multiple assumptions.
\begin{mathpar}

% literals

\ir{S-Bool-False}{\ }{\matches{\falset}{\tbool}}

\ir{S-Bool-True}{\ }{\matches{\truet}{\tbool}}

\ir{S-Num-n}{\ }{\matches{n}{\tnum}}

\ir{S-String-s}{\ }{\matches{s}{\tstr}}

\ir{S-Literal}{\ }{\matches{literal}{literal}}

% quantifiers

\ir{S-LessThan}{n < X}{\matches{n}{\plt{X}}}

\ir{S-GreaterThan}{n > X}{\matches{n}{\pgt{X}}}

\ir{S-EqualTo}{\ }{\matches{n}{\peq{n}}}

% boolean logic

\ir{S-Or-TrueWhatever}{\matches{j}{expr_1}}{\matches{j}{expr_1 \lor expr_2}}

\ir{S-Or-WhateverTrue}{\matches{j}{expr_2}}{\matches{j}{expr_1 \lor expr_2}}

\ir{S-And-TrueTrue}{\matches{j}{expr_1} \\ \matches{j}{expr_2}}{\matches{j}{expr_1 \land expr_2}}

% JSON structures

\ir{S-Array}{\forall i = 0 \ldots |j|-1 \ . \ \matches{j_i}{\tau}}{\matches{\jarr{j^*}}{\jarr{\tau}}}

% Not sure if this is the right way to notate it, but it's supposed to say that a matched key and a matched value lead to a matched key-value pair
\ir{S-KeyValue}{\matches{s'}{s} \\ \matches{j}{\tau}}{\matches{j_{s'}}{s: \tau}}

% matching a key-value list gives you an object
\ir{S-Obj}
  {\forall s' \in s \ . \ \matches{s':j_{s'}}{kv}}
  {\matches{\jobj{(s: j)^*}}{\jobj{kvl}}}


\end{mathpar}

\problem{Problem 2}

Part 1:

\begin{mathpar}

\ir{S-Array}{\ }{\steps{([n]\ a, \jarr{\ldots,j_n,\ldots})}{(a, j_n)}}

\ir{S-Property}{\ }{\steps{(\akey{s}{a}, \jobj{s': j',\ldots})}{(a, j')}}

\ir{S-Map}{\forall i=0 \ldots {j}-1 \ . \ \steps{(a, j_i)}{(b, j'_i)}}{\steps{(\apipe{a}, \jarr{j^*})}{(b, \jarr{j'^*})}}

% Not sure if this is needed, but the instructions said there should be 4 rules, so 🤷

\ir{S-Terminate}{\steps{(a\ b, j)}{(b, j')}}{\steps{(a, j)}{(\aeps, j)}}

\end{mathpar}

Part 2:

\begin{mathpar}

% base case: epsilon always matches; it extracts the current JSON value
\ir{S-BaseCase}{\ }{\matches{\aeps}{\tau}}

% pre-conditions are the same for these two
\ir{S-Map}{\matches{a}{\tau}}{\matches{\apipe{a}}{\jarr{\tau}}}
\ir{S-Index}{\matches{a}{\tau}}{\matches{\aidx{n}{a}}{\jarr{\tau}}}

% Only have to match one key-value in an object schema
\ir{S-Key}{\exists (s', \tau') \in \jobj{(s:\tau)^*} \ . \ s' = s \land \matches{a}{\tau'}}{\matches{\akey{s}{a}}{\jobj{(s:\tau)^*}}}


\end{mathpar}

\textit{Accessor safety}: for all $a, j, \tau$, if $\matches{a}{\tau}$ and $\matches{j}{\tau}$, then there exists a $j'$ such that $\evals{(a, j)}{\aeps, j'}$.

\begin{proof}
% Proof goes here.
\end{proof}

\end{document}
